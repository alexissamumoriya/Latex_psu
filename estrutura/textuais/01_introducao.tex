% INTRODUÇÃO-------------------------------------------------------------------

\chapter{INTRODUÇÃO}
\label{chap:introducao}

O estudo do equilíbrio de fases é de grande interesse nos processos químicos e em várias áreas. Informações sobre o equilíbrio sólido-líquido de misturas de ácidos graxos, ácido graxo e álcool, e álcoois são de interesse científico e industrial.  Nas indústrias de alimentos os procedimentos de emulsificação, aeração e técnicas de processamentos em alta pressão,  os quais tem como objetivo proporcionar várias texturas sem alterar o valor nutritivo, demandam por esses estudos. No setor do biodiesel, a determinação do \textit{cloud point} (ponto de nuvem ou ponto de turvação), indica a capacidade do biodiesel não solidificar durante seu uso como combustível. Na industria de cosmético está relacionado às emulsões. \cite{Leggieri2018a,Costa2007,Costa2009,Rocha2009a,Prausnitz,Rocha2011}

Prever quais fases e suas composições é  muito relevante em vários processos e operações da indústria. Essas fases podem ser  determinadas por meio de cálculos de equilíbrio de fase e análises de estabilidade. Os comportamentos do equilíbrio de fase para misturas sólido-líquido podem ser determinados, por calorimetria exploratória diferencial (DSC) ou por cálculos com modelos matemáticos. A aplicação de  modelos termodinâmicos químicos, como de  \textit{Wilson, Nonrandom Two Liquid Theory} (NRTL), Margules e \textit{Universal Quasi-chemical Theory (UNIQUAC)}, entre outros, tem o intuito de determinar os coeficientes de variedade. Dessa forma, os cálculos de equilíbrio de fase tem como objetivo, pela  minimização global da energia livre de \textit{Gibbs} com técnicas de otimização global, buscar e determinar pontos de transição da fase para a construção de seus respectivos diagramas para obtenção das linhas referentes a essas transições. \cite{Muller2019,Costa2009,Rocha2011,Farajnezhad2016}

 Portanto, esse trabalho propõe modelagens termodinâmicas para descrever o equilíbrio sólido-líquido de misturas binárias de ácidos carboxílicos e ou álcoois de  cadeias carbônicas longas, sendo então verificada a aplicação dos modelos de Margules, no formato Simétrico e Assimétrico, e Wilson. Uma melhor compreensão  desses equilíbrios  pode colaborar com o desenvolvimento  de  melhores  produtos,  princialmente,   o biodiesel que é uma alternativa, ambientalmente viável,  ao diesel.
