% JUSTIFICATIVA------------------------------------------------------------------

\chapter{JUSTIFICATIVA}
\label{chap:justifiativa}

Um dos grandes interesses em compreender o equilíbrio de fase sólido-líquido para misturas graxas, é o estudo termodinâmico com a finalidade de estabelecer um modelo de cristalização para biodiesel, pois o biodiesel em baixas temperatura possui um ponto de turvação ou ponto de cristalização, chamado de \textit{Cloud Point}. Este ponto tem como característica a ocorrência da formação de cristais e como consequência, a menor fluidez do componente, o que não ocorre com combustível derivado do petróleo, inviabilizando sua comercialização.

Um dos pontos importantes é escolha das matérias-primas utilizadas na fabricação do biodiesel, as quais influenciam as propriedades do mesmo, segundo Dias (\citeyear{Angelica}), o óleo de Soja possui de 49,7\% até 56,9\% de Ácido Linoléico e de 9,9\% até 12,2\% de Ácido Palmítico em sua composição; o Óleo de Rícino possui de 2,9\% até 6,5\% de Ácido Linoléico e de 1,4\% até 2,1\% de Ácido Esteárico e o Óleo de Crambe possui 3,4\% de Ácido Palmítico e de 2,9\% até 6,5\% de Ácido Linoléico, dessa forma, diante diferenças de composições, estudar o comportamento das misturas binárias e seu comportamento termodinâmico tem sua relevância.

Ainda referente as composições, se a mistura possuir a fração elevada de  ácidos graxos insaturados, o combustível é mais vulnerável à oxidação;  tem pior capacidade de armazenamento,  menos fluidez no clima frio e menor estabilidade da oxidação. Estas características tornam mais difícil comercializar o biodiesel obtido de gorduras animais, que são mais insaturadas.  \cite{Leggieri2018a,Costa2012,Imahara2006}

%Os álcoois graxos,  que possuem entre 6 e 22 carbonos em sua cadeia, podem ser  extraídos  de plantas.  Essas molécula apresentam uma cadeia carbônica longa,  hidrofóbica,  e uma hidroxila, que permite que esses álcoois formem ligações de hidrogênio. Os álcoois graxos possuem variadas aplicações  industriais nos cosméticos suas aplicações  são como controladores de viscosidade, formadores de emulsões e emolientes; nos alimentos o uso desses álcoois resulta em diferentes texturas, nos fármacos desempenham um papel importante na temperatura de  fusão e solubilidade. \cite{Barbosa2012}

Aplicar modelos matemáticos a problemas reais pode auxiliar a tomada de decisão, contorna fases de teste diminuindo a formação de resíduos, padroniza a fabricação de produtos, além de ajudar na compreensão de fenômenos. As técnicas experimentais são vastamente utilizadas e confiáveis, como a técnica de Calorimetria Exploratória Diferencial (DSC), espectroscopia infravermelho com transformada de \textit{Fourier, Raman} ou difração de raio X; no entanto, utilizar modelos matemáticos tem suas vantagens, visto que, não necessita destes equipamentos e vai ao encontro dos princípios da química verde, que defende a utilização do mínimo de material logo minimiza os resíduos.\cite{DeMarco2019,Prausnitz,Goulart2019}

Um melhor entendimento sobre os equilíbrios sólido-líquidos das misturas graxas pode favorecer a produção de cosméticos, alimentos, fármacos e principalmente, biodiesel. \cite{Rocha2011,Leggieri2018a,Costa2007,Wei2009,Boudouh2016,Costa2012,Costa2009}

Esse trabalho utilizou de cálculos teóricos computacionais para o estudo do equilíbrio sólido-líquido de misturas de álcoois e ou ácidos graxos, permitindo um melhor entendimento do comportamento dessas misturas e sugere a possibilidade de utilização desses compostos, que podem ter origem vegetal em substituição a outros de origem fóssil.
