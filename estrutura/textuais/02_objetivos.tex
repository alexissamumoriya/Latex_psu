%OBJETIVOS--------------------------------------------------------

\chapter{OBJETIVOS}
\label{chap:objetivos}

\section{Objetivo Geral}

O objetivo deste trabalho é utilizar modelos termodinâmicos capazes de indicar parâmetros para o equilíbrio de fase e determinar o diagrama de fase sólido-líquido por meio de modelagem matemática com recursos computacionais, com a finalidade de encontrar o \textit{Cloud Point}, ponto de turvação ou cristalização do biodiesel, além das outras aplicações industriais, com as misturas binárias encontradas na literatura com técnicas de DSC.

\section{Objetivo Específico}

Identificar quais os ácidos graxos e álcoois que são encontrados no biodiesel e quais são as proporções.

Buscar sistemas binários com métodos de DSC (Calorimetria exploratória diferencial) com misturas dos ácidos graxos e/ou álcoois, cuja a finalidade é implementar dos modelos termodinâmicos de Margules Assimétrico (MA), Margules Simétrico (MS) e Wilson.

Construir os diagramas de fase para cada método de DSC e comparar os valores encontrado com o Coeficiente de Determinação ou $R^2$.

Comparar os diagramas de fases dados experimentais via DSC com os calculados pelos modelos termodinâmicos.