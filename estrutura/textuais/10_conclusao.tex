% CONCLUSÃO--------------------------------------------------------------------

\chapter{CONSIDERAÇÕES FINAIS}
\label{chap:conclusao}

De acordo com os objetivos propostos, foram definidos os compostos carbônicos capazes de compor as misturas graxas e/ou alcoólicas, com ácidos graxos e álcoois, para que haja a identificação e definição dos diagramas de fases, utilizando a modelagem matemática e termodinâmica que calculam e encontram os critérios que identificam do equilíbrio de fase sólido-liquido. De tal forma que o uso teórico e computacional é vital para o trabalho, para aplicação dos métodos e resultados sustentáveis.

Após determinação dos sistemas a serem estudados, tanto para misturas binárias foi possível calcular os parâmetros termodinâmicos necessários e característicos de cada modelo usado, baseado na busca pelo mínimo global da energia livre de Gibbs, para definição das fases sólido-liquido, da linha \textit{liquidus} e dos pontos eutéticos ou peritéticos. Os modelos termodinâmico junto a modelagem matemática, indicam de forma sensata a linha \textit{liquidus} que  representa a transição de fase sólido-liquido, as outras transições intermediárias apresentadas pelas técnicas experimentais disponíveis, apresentam maiores dificuldades de representação teórico/computacional.

Como parte de grande relevância, ainda ressalta-se que, a partir das comparações com termogramas experimentais, a identificar dos pontos de transições característicos do ESL, ou seja, definição de ponto eutético ou peritético garantem a qualidade do trabalho desenvolvido.

Diante dos resultados obtidos pelos modelos de Margules Assimétrico, Margules Simétricos e Wilson que buscam resultados próximos da técnica de DSC, para observar essa aproximação dos resultados, foi usada o coeficiente de determinação ou $R^2$ com a característica de verificar a representatividade dos modelo termodinâmicos, na intenção viabilizar o uso em industrias de biodiesel com o objetivo de determinar o ponto de cristalização com determinadas misturas de ácidos graxos com ácidos graxo e/ou álcoois. Também pode ser aplicados em outros ramos da industrias com a meta minimizar resíduos químicos, contornando etapas da industrialização. E não menos importante é capacidade de prever o comportamento de um determinado produto com a variação de temperatura.

\chapter{TRABALHOS FUTUROS}
\label{sec:trabalhosFuturos}

O trabalho foi exclusivamente aplicado em sistemas de misturas binárias, levando em consideração a aplicabilidade em biodiesel, industrias de cosméticos e alimentos, o estudo pode ser estendido para sistema com misturas ternárias e ter aplicação mais abrangente para outros grupamentos funcionais de cadeias longas com características e comportamentos semelhantes. 
Outra opção para futuras aplicações são escolhas de sistemas graxos específicos diante percentuais evidentes de possibilidades de óleos com grande potencial para produção de biodiesel, como é o caso do óleo de Rício, cuja composição de ácido ricinoléico está entre 84,0\% até 91,0\%, fazendo desta experimentação teórico/computacional mais favorável a tais desenvolvimentos. 

