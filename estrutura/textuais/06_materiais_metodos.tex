\chapter{MATERIAIS E MÉTODOS}
\label{Material_metodo}

\section{Materiais}
\hspace{0.5cm} O trabalho tem caráter teórico e computacional com aplicação de técnicas de otimização matemática de programação não-linear, implementado no \textit{software} $GAMS$ (\textit{General Algebric Model System}), planilha eletrônica, \textit{software} Geogebra, \textit{PyCharm}.

\section{Métodos}

\hspace{0.5cm} Para o determinar o equilíbrio de fase sólido-líquido, foram aplicados os modelos termodinâmicos, por meio de cálculos pode-se determinar coeficientes ligados diretamente com a Energia livre de Gibbs.  \cite{Rocha2009a,Costa2007}

Segundo ROCHA \citeyear{Rocha2009a}, o diagrama de fase sólido-líquido para misturas binárias pode ser encontrado pela minimização da energia de Gibbs. O método para determinar o diagrama de fase sólido-líquido é capaz de incluir o número de componentes ($NC$) desejado, o valor de componentes para esse trabalho é de $NC=2$, com possibilidade futura de aplicações em misturas de mais componentes. A condição necessária para o mínimo, de componentes e produto, é dada pela convexidade do problema. Mas uma suposição implícita usada aqui no modelo é de que existe apenas uma fase líquida. Portanto para um modelo geral de atividade líquida deve ter uma análise cuidadosa, pois em modelos não-convexos, podem ser verificadas existências de outras possibilidades de equilíbrio, do tipo líquido-líquido-sólido, por exemplo. 

\subsection{Modelo geral para fase líquida}

\hspace{0.5cm} A equação abaixo indicada por $B_{i}$ (\ref{eq:Fugacidade_1}) é o parâmetro relativo a fugacidade nas fases líquida e sólida do componente $i$, em que $\Delta h_{fi}$ representa entalpia de fusão para o componente $i$, $T_{fi}$ é a temperatura de fusão do componente $i$.
\begin{equation}\label{eq:Fugacidade_1}
B_{i}=\frac{\Delta h_{fi}}{R\cdot T_{fi}}\cdot\left(\frac{T_{fi}}{T}-1\right)
\end{equation}

Logo é possível determinar expressões para a temperatura em função da fração molar, que na maioria dos modelos líquidos exigirá um procedimento interativo.

Existem três regiões possíveis no diagrama de fases $T\times x_{1}$ para o ESL:  Região I é aquela em que a fase sólida é o componente 2 e representada pela equação e a respectiva restrição (Eq. \ref{eq:Regiao_I_1} e \ref{eq:Regiao_I_2}); A região II é aquela em que a fase sólida é o componente 1 e representada pelas equação e a respectiva restrição (Eq. \ref{eq:Regiao_II_1} e \ref{eq:Regiao_II_2}); A região III, se existir, é aquela em que a fase sólida é o composto  e representada pelas equação e respectivas restrições (Eqs. \ref{eq:Regiao_III_1}, \ref{eq:Regiao_III_2} e \ref{eq:Regiao_III_3}).

\subsubsection{Região I}
\begin{equation}\label{eq:Regiao_I_1}
T=\frac{\Delta h_{f2}}{\frac{\Delta h_{f2}}{T_{f2}}-R\cdot\ln[(1-x_{1})\gamma_{2}^{l}]}
\end{equation}
limitado pelas restrições:
\begin{equation}\label{eq:Regiao_I_2}
-R\cdot T\cdot[B_{1}+\ln x_{1}]-R\cdot T\cdot\ln\gamma_{1}^{l}\geq\max\left\{0,\frac{1}{v_1}\cdot\Delta G_{R}^{\circ} \right\}
\end{equation}

\subsubsection{Região II}
\begin{equation}\label{eq:Regiao_II_1}
T=\frac{\Delta h_{f1}}{\frac{\Delta h_{f1}}{T_{f1}}-R\cdot\ln[(x_{1})\gamma_{1}^{l}]}
\end{equation}
limitado pelas restrições:
\begin{equation}\label{eq:Regiao_II_2}
-R\cdot T\cdot[B_{2}+\ln(1- x_{1})]-R\cdot T\cdot\ln\gamma_{2}^{l}\geq\max\left\{0,\frac{1}{v_2}\cdot\Delta G_{R}^{\circ} \right\}
\end{equation}

\subsubsection{Região III}
\begin{equation}\label{eq:Regiao_III_1}
T=\frac{\nu_{1}\cdot\Delta h_{f1} + \nu_{2}\cdot\Delta h_{f2} + \Delta G_{R}^{\circ}}{\nu_{1}\cdot\frac{\Delta h_{f1}}{T_{f1}}+\nu_{2}\cdot\frac{\Delta h_{f2}}{T_{f2}}-\nu_{1}\cdot R\cdot\ln[(x_{1})\gamma_{1}^{l}]-\nu_{2}\cdot R\cdot\ln[(1-x_{1})\gamma_{2}^{l}]}
\end{equation}
limitado pelas restrições:
\begin{equation}\label{eq:Regiao_III_2}
0\leq-R\cdot T\cdot[B_{1}+\ln x_{1}]-R\cdot T\cdot \ln\gamma_{1}^{l}\leq\max\left\{0,\frac{1}{\nu_1}\cdot
\Delta G_{R}^{\circ} \right\}
\end{equation}
\begin{equation}\label{eq:Regiao_III_3}
0\leq-R\cdot T\cdot[B_{2}+\ln(1- x_{1})]-R\cdot T\cdot\ln\gamma_{2}^{l}\leq\max\left\{0,\frac{1}{\nu_2}\cdot
\Delta G_{R}^{\circ} \right\}
\end{equation}

As equações  não são realmente explícitos em $T$, pois $\gamma_{i}^{l}$ são em função de $T$ e $x_{i}^{l}$.

\subsection{Modelo de Margules Simétrico para fase líquida}

Será usado nesse trabalho a fase líquida modelada pela equação (Eq. \ref{eq:Margules_1}) considerando o modelo de Margules com 2-sufixo:
\begin{equation}\label{eq:Margules_1}
R\cdot T\cdot\ln\gamma_{i}^{l}=\frac{1}{2} \cdot\sum_{j=1}^{NC}\sum_{k=1}^{NC}\left( A_{ij}+A_{ji}-A_{jk}\right) \cdot x_{j}^{l}\cdot x_{k}^{l}
\end{equation}

Considerando que $A_{ii}=0$ e $A_{ij}=A_{ji}$, a energia livre de Gibbs para problemas com uma mistura binária em uma fase líquida pode-se escrever:

\begin{equation}
\begin{split}
G=(\eta_{1}^{0}\cdot\nu_{1}^{S,\circ}+\eta_{2}^{0}\cdot\nu_{2}^{S,\circ})+ \eta_{p}^{0}\cdot\Delta G_{R}^{\circ}+ \eta_{p}^{S}\cdot\Delta G_{R}^{\circ} +R\cdot T\\\cdot \left[\eta_{1}^{l} \left( B_{1}+\ln\eta_{1}^{l}-\ln(\eta_{1}^{l} +\eta_{2}^{l} )\right)+\eta_{2}^{l}\left( B_{2}+\ln\eta_{2}^{l}-\ln(\eta_{1}^{l} +\eta_{2}^{l} )\right) \right]
\\+\frac{A_{12}\eta_{1}^{l}\eta_{2}^{l}}{\eta_{1}^{l}+\eta_{2}^{l}}
\end{split}
\end{equation}

Conforme o formulado, possui apenas um ponto mínimo se o problema for restrito a $\dfrac{A_{12}}{R}\cdot T\leq2$, de forma que a condição de Kutn-Tucker sejam necessárias e suficientes para o mínimo global \cite{Edgar2001}.

Dessa forma usando as equação (\ref{eq:Regiao_I_1}), (\ref{eq:Regiao_II_1}), (\ref{eq:Regiao_III_1}) e (\ref{eq:Margules_1}), determinam as equações explícitas para a temperatura $T$, em função da fração molar do componente 1 da fase $x_ 1$.

\subsubsection{Região I}

\begin{equation}
T=\frac{\Delta h_{f2}+A_{12}\cdot x_{1}^{2}}{\frac{\Delta h_{f2}}{T_{f2}} -R\cdot\ln(1-x_{1})}
\label{equa_5}
\end{equation}

limitado pelas restrições:

\begin{equation}
-R\cdot T\cdot[B_{1}+\ln x_{1}]-A_{12}\cdot(1-x_{1})^{2}\geq\max\left\{0,\frac{1}{\nu_1}\cdot\Delta G_{R}^{\circ} \right\}
\label{limite_5}
\end{equation}

\subsubsection{Região II}

\begin{equation}
T=\frac{\Delta h_{f1}+A_{12}\cdot (1-x_{1})^{2}}{\frac{\Delta h_{f1}}{T_{f1}} -R\cdot\ln x_{1}}
\label{equa_6}
\end{equation}

limitado pelas restrições:

\begin{equation}
-R\cdot T\cdot[B_{2}+\ln (1-x_{1})]-A_{12}\cdot x_{1}^{2}\geq\max\left\{0,\frac{1}{\nu_2}\cdot\Delta G_{R}^{\circ} \right\}
\label{limite_6}
\end{equation}

\subsubsection{Região III}

\begin{equation}
T=\frac{\nu_{1}\cdot\Delta h_{f1} + \nu_{2}\cdot\Delta h_{f2}+\nu_{1}\cdot A_{12}\cdot(1-x_{1})^{2}+ \nu_{2}\cdot A_{12}\cdot x_{1}^{2} + \Delta G_{R}^{\circ}}{\nu_{1}\cdot\frac{\Delta h_{f1}}{T_{f1}}+\nu_{2}\cdot\frac{\Delta h_{f2}}{T_{f2}}-\nu_{1}\cdot R\cdot\ln x_{1}-\nu_{2}\cdot R\cdot\ln(1-x_{1})}
\label{equa_7}
\end{equation}
limitado pelas restrições:
\begin{equation}
0\leq-R\cdot T\cdot[B_{1}+\ln x_{1}]-A_{12}\cdot (1+x_{1})^{2}\leq\max\left\{0,\frac{1}{\nu_1}\cdot
\Delta G_{R}^{\circ} \right\}
\label{limite_7}
\end{equation}
\begin{equation}
0\leq-R\cdot T\cdot[B_{2}+\ln(1- x_{1})]-A_{12}\cdot x_{1}^{2}\leq\max\left\{0,\frac{1}{\nu_2}\cdot
\Delta G_{R}^{\circ} \right\}
\label{limite_8}
\end{equation}

\subsection{Modelo de Margules Simétrico}

Para o desenvolvimento do algoritmo para a transição de fase, considere a fase sólida como a ideal que é determinada por SLAUGHTER e
DOHERTY \citeyear{Douglas1995} e a fase liquida modelada representada por \textit{Margules}  2-sufixo.\cite{Rocha2009a}

Para isso a função fugacidade é definida por:
\begin{equation}\label{eq:Margules_3}
d\mu_{i}(T,P)=d\overline{G}_i=RTd\ln(\hat{f}_i)
\end{equation}
em que $T$ é uma constante e aplicando a integração, obtemos
\begin{equation}\label{eq:Margules_4}
\mu_{i}(T,P)=\mu^{\circ}_{i}(T)+RT\ln\left(\dfrac{\hat{f}_i}{f^{\circ}_{i}}\right)
\end{equation}
para o gás ideal a 1$atm$, obtemos $f^{\circ}_{i}=1$ e para um composto puro $f^{\circ}_{i}=f^{\circ}_{\ell,i}$. \cite{Rocha2009a}

Energia Livre de \textit{Gibbs} total do sistema e dada por
\begin{equation}\label{eq:Gibbs_1}
G=\sum_{k=1}^{NF}\sum_{i=1}^{NC}\eta_{i}^{k}\cdot\mu_{i}^{k}
\end{equation}
substituindo a equação \ref{eq:Margules_4} em \ref{eq:Gibbs_1}, obtém:
\begin{equation}\label{eq:Gibbs_2}
G=\sum_{k=1}^{NF}\sum_{i=1}^{NC}\eta_{i}^{k}\cdot\mu_{i}^{\circ}+RT\sum_{k=1}^{NF}\sum_{i=1}^{NC}\eta_{i}^{k}\ln\left(\dfrac{\hat{f}_i}{f^{\circ}_{i}}\right)
\end{equation}
minimizando $G$ em relação à $\eta_{i}^{k}$ em que $P$ e $T$ são constante, isto é
\begin{equation}\label{eq:Gibbs_3}
G=\sum_{k=1}^{NF}\sum_{i=1}^{NC}\eta_{i}^{k}\ln\left(\dfrac{\hat{f}_i}{f^{\circ}_{i}}\right)
\end{equation}
\cite{Rocha2009a}.

Será usado nesse trabalho a fase líquida modelada pela equação (Eq. \ref{eq:Margules_2}) considerando o modelo de Margules com 2-sufixo:
\begin{equation}\label{eq:Fugacidade_2}
\hat{f}_i^{k}=f_{i}^{\circ}x_i^{k}\gamma_i^{k}
\end{equation}
em que $f_{i}^{\circ}$ representa a fugacidade na componete $i$ para composto puro e na fase $k$.

\subsection{Margules 2-sufixos para fase liquida}

Para a fase líquida, usamos o modelo de \textit{Margules} Assimétrico
\begin{equation}
\underline{G}^{ex}=RT\sum_{i-1}^{NC}x_i\ln(\gamma_{i})=\sum_{k=1}^{NF}\sum_{i=1}^{NC}A_{ij}x_ix_j
\end{equation}
para implementação no \textit{GAMS}
\begin{equation}\label{eq:margules_6}
G=\sum_{i=1}^{NC}\left[\eta_{i}^{\ell}\left(B_i+\ln\eta_i^\ell -\ln\sum_{j=1}^{NC}\eta_{i}^\ell+\dfrac{\displaystyle\sum_{j=1}^{NC}A_{ij} \eta_{j}^\ell}{2RT\displaystyle\sum_{i=1}^{NC}\eta_{i}^\ell}\right)\right]
\end{equation}
\cite{Rocha2009a}.

E \textit{Margules} Simétrico para o \textit{GAMS} usa \textit{Margules} 2-sufixo em que $A_{ii}=0$ e $A_{ij}=A_{ji}$ em a equação \ref{eq:margules_6} é dada por
\begin{equation}
G=\sum_{i=1}^{NC}\left[\eta_{i}^{\ell}\left(B_i+\ln\eta_i^\ell -\ln\sum_{j=1}^{NC}\eta_{i}^\ell\right)\right]+\dfrac{1}{RT}\dfrac{\displaystyle\sum_{i<j}A_{ij} \eta_{i}^\ell  \eta_{j}^\ell }{2RT\displaystyle\sum_{i=1}^{NC}\eta_{i}^\ell}
\end{equation}
em que $B_{i}$ é o parâmetro do modelo para determinar o mínimo de \textit{Gibbs}.\cite{Rocha2009a}

\subsection{Wilson}
A equação de \textit{Wilson} é representada por
\begin{equation}
\ln\gamma_{i}=-\ln(x_i+x_j\Lambda_{ij})+x_j\left[ \dfrac{\Lambda_{ij}}{x_i+x_j\Lambda_{ij}}-\dfrac{\Lambda_{ji}}{x_i\Lambda_{ji}+x_j}\right]
\end{equation}
\begin{equation}
\ln\gamma_{j}=-\ln(x_j+x_i\Lambda_{ji})+x_j\left[ \dfrac{\Lambda_{ji}}{x_j+x_i\Lambda_{ji}}-\dfrac{\Lambda_{ij}}{x_i\Lambda_{ij}+x_j}\right]
\end{equation}

\subsection{Ponto eutético e ponto peritético}

Como $\nu_{1}<0$ e $\nu_{2}<0$ se $\Delta G_{R}^{\circ}$ a Região III não existe. Neste caso, seja encontrada a composição do ponto eutético é determinado pela intersecção das equações (\ref{equa_5}) e (\ref{equa_6}):

\begin{equation}
\frac{\Delta h_{f2}+A_{12}\cdot x_{1}^{2}}{\frac{\Delta h_{f2}}{T_{f2}} -R\cdot\ln(1-x_{1})}=\frac{\Delta h_{f1}+A_{12}\cdot (1-x_{1})^{2}}{\frac{\Delta h_{f1}}{T_{f1}} -R\cdot\ln x_{1}}
\end{equation}

onde cada uma tem as restrições (\ref{limite_5}) e (\ref{limite_6}).

Do mesmo modo $\nu_{1}<0$ e $\nu_{2}<0$, se $\Delta G_{G}^{\circ}<0$ a Região III existe, de modo que existem dois pontos:

\begin{equation}
\frac{\Delta h_{f2}+A_{12}\cdot x_{1}^{2}}{\frac{\Delta h_{f2}}{T_{f2}} -R\cdot\ln(1-x_{1})}=\frac{\nu_{1}\cdot\Delta h_{f1} + \nu_{2}\cdot\Delta h_{f2}+\nu_{1}\cdot A_{12}\cdot(1-x_{1})^{2}+ \nu_{2}\cdot A_{12}\cdot x_{1}^{2} + \Delta G_{R}^{\circ}}{\nu_{1}\cdot\frac{\Delta h_{f1}}{T_{f1}}+\nu_{2}\cdot\frac{\Delta h_{f2}}{T_{f2}}-\nu_{1}\cdot R\cdot\ln x_{1}-\nu_{2}\cdot R\cdot\ln(1-x_{1})}
\label{equa_8}
\end{equation}
e 
\begin{equation}
\frac{\Delta h_{f1}+A_{12}\cdot (1-x_{1})^{2}}{\frac{\Delta h_{f1}}{T_{f1}} -R\cdot\ln x_{1}}=\frac{\nu_{1}\cdot\Delta h_{f1} + \nu_{2}\cdot\Delta h_{f2}+\nu_{1}\cdot A_{12}\cdot(1-x_{1})^{2}+ \nu_{2}\cdot A_{12}\cdot x_{1}^{2} + \Delta G_{R}^{\circ}}{\nu_{1}\cdot\frac{\Delta h_{f1}}{T_{f1}}+\nu_{2}\cdot\frac{\Delta h_{f2}}{T_{f2}}-\nu_{1}\cdot R\cdot\ln x_{1}-\nu_{2}\cdot R\cdot\ln(1-x_{1})}
\label{equa_10}
\end{equation}

Onde $T$ mais baixo é o ponto eutético, enquanto o outro é o ponto peritético.

A equação (\ref{equa_8}) pode ser reorganizada para:

\begin{equation}
\frac{\Delta h_{f2}+A_{12}\cdot x_{1}^{2}}{\frac{\Delta h_{f2}}{T_{f2}} -R\cdot\ln(1-x_{1})}=\frac{\Delta h_{f1}+A_{12}\cdot(1-x_{1})^{2} +\frac{1}{\nu_{1}}\cdot\Delta G_{R}^{\circ}}{\frac{\Delta h_{f2}}{T_{f2}} -R\cdot\ln (1-x_{1})}
\label{equa_9}
\end{equation}

a equação (\ref{equa_9}) possui os mesmos limites de restrições de (\ref{limite_5}) e (\ref{limite_7}).

E reescrevendo a equação (\ref{equa_10}):
\begin{equation}
\frac{\Delta h_{f1}+A_{12}\cdot (1-x_{1})^{2}}{\frac{\Delta h_{f1}}{T_{f1}} -R\cdot\ln(x_{1})}=\frac{\Delta h_{f2}+A_{12}\cdot x_{1}^{2} +\frac{1}{\nu_{2}}\cdot\Delta G_{R}^{\circ}}{\frac{\Delta h_{f2}}{T_{f2}} -R\cdot\ln (1-x_{1})}
\label{equa_11}
\end{equation}

em que a equação (\ref{equa_11}) tem as mesmas restrições de (\ref{limite_6}) e (\ref{limite_8}).

\subsection{Aplicações das ferramentas computacionais}
\subsubsection{Geogebra}

Neste trabalho foi utilizado \textit{Geogebra Clássico 5}  uma versão \textit{offline}, com uma interface mais pragmática que traz maior confiança no trabalho, Figura \ref{fig:12}, o motivo de usar essa ferramenta para coleta de dados de DSC nos artigos pesquisados, já que determinados artigos não possuem os dados em forma de tabela.
\begin{figure}[H]
	\centering
	\includegraphics[width=0.9\linewidth 
	%,height=0.4\textheight
	]{dados/figuras/Geogebra_2.png}
	\caption[Interface gráfica do Geogebra]{Interface gráfica do Geogebra}
	\label{fig:12}
\end{figure}

A finalidade do Geogebra tem o propósito a coleta dos pares ordenados  (\textit{misturas;temperatura}), já que alguns dados de misturas e temperaturas não são fornecidas em forma de tabela, como para o sistema hexadecanol e ácido mirístico, a grande preocupação de coletar os dados a partir do diagrama é adquirir esse valores de forma precisa com base nos artigos, o procedimento para obtenção dos dados segue os passos: 
\begin{itemize}
    \item Tirar a foto com a tecla \textit{print screen} ou \keystroke{PrtScr} (a simplificação dos teclados pode variar muito de marca e modelo de teclado), para agilizar é de bom senso usar as tecla de atalho \Shift$+$\keystroke{PrtScr} para capturar apenas a área desejada e manter a qualidade da imagem.
    \item Utilizar um aplicativo de imagem para alterar escala do eixos das abscissa e ordenada, para que os eixos das coordenadas do Geogebra coincida com os eixos das coordenadas do artigo, com a finalidade de que os pares ordenados sejam mais precisa. Para esse trabalho o aplicativo \textit{Kolourpaint}\footnote{Se encontra no site \url{https://apps.kde.org/pt-br/kolourpaint/}, até o momento esse versão é para versões Linux.} foi utilizado pela interface gráfica simples, fácil de se manusear e de código aberto.
    \item Na Figura \ref{fig:DSC_Hexadec+Miristi} mostra o diagrama de fase no Geogebra como imagem de fundo, para isso na \textbf{barra de ferramentas} clique em \textbf{Editar}, \textbf{Inserir imagem de} e \textbf{Arquivo}, ajustar a imagem conforme a escala dos eixos das coordenadas, depois vá em \textbf{Exibir} marque a opção \textbf{Planilha}, desse modo podemos coletar a informação em uma planilha.
    \item Nas coordenadas foi usada uma aproximação de 10 casas decimais, para modificar vá em \textbf{barra de ferramentas} clique em \textbf{Arredondamento} e selecione a aproximação que melhor convém.
    \item Para a coleta automática dos pares ordenados, primeiro inserir todos os pontos e posiciona-los nas respectivas posições da mistura e temperatura, segundo na \textbf{barra de ferramentas} da planilha selecione \textbf{Lista de pontos}, terceiro criada a lista de pontos na \textbf{Janela de Álgebra} clique com botão direito do mouse na lista e vá em \textbf{propriedades} e em \textbf{Definição} adicione os pontos, em quarto clique com botão direito do mouse e selecione \textbf{Gravar para Planilha de Cálculos} e finalmente na planilha clique sobre o ponto vermelho (\textit{rec}) para gravar os valores.
\end{itemize}
\begin{figure}[H]
	\centering
	\includegraphics[scale=0.3]{DSC_Hexadec+Miristi}
	\caption[Geogebra coleta de pares ordenados]{Coleta de pares ordenados}
	\label{fig:DSC_Hexadec+Miristi}
\end{figure}
Conforme a Figura \ref{fig:DSC_Hexadec+Miristi} com todos os procedimentos acima, obtém-se na coluna à esquerda os pares ordenados e na coluna à direita os dados disponíveis na planilha eletrônica.

Recomenda-se paciência no ajuste da escala, pois esse ajuste da imagem com a escala é um procedimento tedioso, requer paciência, levando em varias vezes o mesmo procedimento.

\subsubsection{Planilha eletrônica}
A planilha eletrônica foi utilizada como banco de dados, para armazenar os pares ordenados da mistura e temperatura do DSC, uma forma de armazenar e organizar os dados como, temperatura de fusão, entalpia, cálculos de conversões de unidades de medidas com Celcius para kelvin e  de kilo joule para quilo, o recurso de transposição de dados, isto é, troca os dados de linhas para colunas  o algoritmo da biblioteca R trabalhar com os dados e armazenar dados de caloria máxima obtida diante o cálculo das Temperaturas das referidas regiões classificadas como I, II e III. 

\subsubsection{\textit{GAMS}}
O \textit{software} \textit{GAMS} foi usado para determinar as temperaturas de fusão com as determinadas misturas binárias para os modelos termodinâmicos Margules Simétrico, Margules Assimétrico e Wilson, com o método matemático de  programação não linear, com o solver CONOPT com solução satisfatória.

\subsubsection{\textit{PyCharm} e \textit{R}}
No desenvolvimento deste trabalho ambos \textit{softwares} foram utilizados para coletar os dados da planilha e plotar os gráficos. O algoritmo foi escrito em \textit{R} e compilado pelo \textit{PyCharm}. Todos os dados de mistura e temperatura se encontram no repositório do \textit{GitHub}, junto com os algoritmos dos diagramas e coeficiente de determinação.
