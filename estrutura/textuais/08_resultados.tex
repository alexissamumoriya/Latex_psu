% RESULTADOS-------------------------------------------------------------------

\chapter{Resultados Esperados}

De acordo com os objetivos propostos, foram definidos os compostos carbônicos capazes de compor as misturas carbônicas, como ácidos graxos e álcoois para que haja a identificação e definição dos diagramas de fases, utilizando a modelagem matemática e termodinâmico que calculam e encontram os critérios que identificam o equilíbrio de fase sólido-liquido. De tal forma que o uso teórico e computacional é vital para o trabalho, para aplicação dos métodos e resultados sustentáveis.

Após determinação dos sistemas a serem estudados, tanto para misturas binárias como para possíveis misturas ternárias, sendo possível calcular os coeficientes de fugacidade e atividade, necessários e característicos de cada modelo termodinâmico usado, baseado na busca pelo mínimo global da energia livre de Gibbs, para definição das fases sólido-liquido e da linha liquidus. Os modelo termodinâmico junto a modelagem matemática, indica de forma sensata a linha \textit{liquidus} que representa a transição de fase sólido-liquido, já a transição de fase sólido $+$ líquido possui determinada dificuldade de expressa-la, mesmo que a escolha das misturas seja de forma adequada afim a induzir o modelo a encontrar as temperaturas da acordo com os dados de DSC como mostra a Figura \ref{fig:5}.

Como parte de grande relevância, ainda espera-se, a partir das comparações com termogramas experimentais, identificar pontos de transições características do ESL, ou seja, definição de ponto eutético ou peritético.

Ainda serão realizadas análises numéricas quantitativas para a comprovação da proximidade desta modelagem realizada comparada aos dados experimentais citados, bem como novos e outros sistemas serão estudados para validação desta modelagem e seu uso pois tenho ideia de aplicar um soma quadrada de erro ou algum outro método numérico comparativo, para comprovar a proximidade não apenas pelos gráficos e além destes sistemas aqui estudados, vamos atrás de outras misturas de ácidos graxos e álcoois

