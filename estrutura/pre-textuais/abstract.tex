% ABSTRACT--------------------------------------------------------------------------------

\begin{resumo}[ABSTRACT]
\begin{SingleSpacing}


The thermodynamic models are used with the purpose of determining, in a satisfactory way, the balance of solid-liquid phase in different solutions. For this work, the concept of mathematical modeling and optimization were applied, with thermodynamic models already defined by binary mixtures of fatty acids and alcohols, in order to obtain the line liquidus and the eutectic point or peritectic point for the phase diagrams studied. Therefore, the characteristic of this work is of theoretical development, computational, with focus on sustainability, by the minimization in the use of input and of residue to the environment; since the analyses of these mixtures of fatty acids and alcohols were all accomplished with computational algorithms, thermodynamic models, mathematical modeling of Nonlinear Programming and phase balance theory. The softwares used in this work were: Gams, in the implementation of the algorithm of the nonlinear programming with thermodynamic models from Margules and Wilson with the solver CONOPT applied to the data analysis of the binary mixtures of fatty acids and alcohols, to find the minimization of the Gibbs free energy from the system; GEOGEBRA to collect the data of the articles of \textit{DSC} (Differential scanning calorimetry); \textit{PyCharm} together with the library of the \textit{R},  in the obtainment of the temperature data and of mixture of the electronic spreadsheet to plot the phases diagrams  and LATEX language as a text editor, its mathematical libraries and TIKZ. The mixtures were submitted to the thermodynamic models of Symmetrical Margules, Asymmetrical Margules and Wilson, configured in explicit equations in Temperature (T), in order to obtain the global minimum from the minimization of the Gibbs free energy. The systems studied were, Myristic Acid with Stearic Acid, Palmitic Acid with Stearic Acid, Hexadecanol with Myristic Acid, Hexadecanol with Tetradecanol, Stearic Acid and Linoleic Acid and Palmitic Acid and Linoleic Acid which phases diagrams were determined and graphically compared to diagrams obtained by available calorimetric techniques in the literature. Quantitative analyses were performed for the proof of proximity between the modeling data and the experimental data quoted. The study of these systems generated information about the behavior of the mixtures of the cosmetic industries interest, food and renewable fuels. The models applied in this work suggest that it is possible to apply the mixtures of fatty acids and alcohols of vegetable origin in these industries contributing to the sustainable development.
\vspace{1cm}
		
\textbf{Keywords}: Solid-Liquid Equilibrium, Free Gibbs Energy, Global Minimization.

\end{SingleSpacing}
\end{resumo}

% OBSERVAÇÕES---------------------------------------------------------------------------
% Altere o texto inserindo o Abstract do seu trabalho.
% Escolha de 3 a 5 palavras ou termos que descrevam bem o seu trabalho 
