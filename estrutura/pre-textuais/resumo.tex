% RESUMO--------------------------------------------------------------------------------

\begin{resumo}[RESUMO]
\begin{SingleSpacing}


Os modelos termodinâmicos são utilizados com o intuito de determinar, de maneira satisfatória, o equilíbrio de fase sólido-líquido em diferentes soluções. Para este trabalho, os conceitos de modelagem matemática e otimização foram aplicados, com modelos termodinâmicos já definidos para misturas binárias de ácidos graxo e álcoois, com o propósito de obter a linha \textit{liquidus} e o(s) ponto eutético ou ponto peritético para os diagramas de fase estudados. Sendo assim, a característica deste trabalho é de desenvolvimento teórico, computacional, com foco na sustentabilidade, pela minimização na utilização de insumos e de resíduo ao meio ambiente; já que as análises dessas misturas de ácidos graxos e de álcoois foram todas realizadas com algoritmos computacionais, modelos termodinâmicos, modelagem matemática de Programação Não-Linear e teoria de equilíbrio de fases. Os softwares utilizados no trabalho foram: Gams, na implementação do algoritmo da programação não-linear com modelos termodinâmicos de Margules e Wilson com o solver CONOPT aplicado a análise dos dados de misturas binárias dos ácidos graxos e álcoois, para encontrar a minimização da energia livre de Gibbs do sistema; GEOGEBRA para coletar o dados de DSC (Calorimetria exploratória diferencial); PyCharm, junto com a biblioteca do R, na obtenção dos dados de temperatura e de mistura da planilha eletrônica para plotar os diagramas de fases e a linguagem \LaTeX como editor de texto, suas bibliotecas matemáticas e TIKZ. As misturas foram submetidas aos modelos termodinâmicos de Margules Simétrico, Margules Assimétrico e Wilson, configurados em equações explícitas em Temperatura ($T$), com intenção de obter o mínimo global a partir da minimização da Energia livre de Gibbs. Os sistemas estudados foram, Ácido Mirístico com Ácido Esteárico, Ácido Palmítico com Ácido Esteárico, Hexadecanol com Ácido Mirístico, Hexadecanol com Tetradecanol, Ácido Esteárico e Ácido Linoleico e Ácido Palmítico e Ácido Linoleico cujos diagramas de fases foram determinados e comparados graficamente com diagramas obtidos por técnicas calorimétricas disponíveis na literatura. Análises  quantitativas foram realizadas para a comprovação da proximidade entre os dados da modelagem e os dados experimentais citados. O estudo destes sistemas gerou informações sobre o comportamento de misturas de interesse das industrias de cosméticos, alimentos e combustíveis renováveis. Os modelos aplicados nesse trabalho sugerem que é possível aplicar as misturas de ácidos graxos e álcoois de origem vegetal nessas industrias contribuindo com o desenvolvimento sustentável. 

\vspace{1cm}

\textbf{Palavras-chave}:Equilíbrio Sólido-Liquido, Energia Livre de \textit{Gibbs}, Minimização Global.

\end{SingleSpacing}
\end{resumo}

% OBSERVAÇÕES---------------------------------------------------------------------------
% Altere o texto inserindo o Resumo do seu trabalho.
% Escolha de 3 a 5 palavras ou termos que descrevam bem o seu trabalho 

